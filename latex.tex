\documentclass[12pt, a4paper]{scrartcl}
\usepackage[utf8]{inputenc}
\usepackage[affil-it]{authblk}
\usepackage{hyperref}
\usepackage{amssymb}
\usepackage{amsmath}
\usepackage{enumitem}
\usepackage{etoolbox}

\begin{document}

\title{Tugas 2 Aljabar Linear 2022}
\author{Hezkiel Bram Setiawan\\M0521030\\dibuat menggunakan \LaTeX\\Source code: \href{https://github.com/haizk/Linear-Transformations-and-Orthogonality}{github.com/haizk}}
\affil{Fakultas Matematika dan Ilmu Pengetahuan Alam\\Universitas Sebelas Maret}
\date{}
\maketitle

\section*{Linear Transformations}
    \begin{enumerate}
        \item \textbf{Exercise 7.1.3}\\In each case, assume that $T$ is a linear transformation.
        \begin{enumerate}
            \setcounter{enumii}{5}
            \item $\mbox{If } T : \textbf{P}_2 \to \mathbb{R} \mbox{ and } T(x+2)=1 \mbox{, } T(1)=5 \mbox{, } T(x^2+x)=0 \mbox{, find } T(2-x+3x^2)$.

            Diketahui:
            \begin{itemize}
                \item[] $T : \textbf{P}_2 \to \mathbb{R}$\item[]$T(x+2)=1$\item[]$T(1)=5$\item[]$T(x^2+x)=0$
            \end{itemize}

            Ditanya:
            \begin{itemize}
                \item Nilai $T(2-x+3x^2)$?
            \end{itemize}

            Jawab:
            \begin{align*}
                T(2-x+3x^2) &= T(3x^2-x+2)\\
                &= T(3x^2+3x-4x-8+10)\\
                &= T(3(x^2+x))-T(4(x+2))+T(10(1))\\
                &= 3T(x^2+x)-4T(x+2)+10T(1)\\
                &= 3(0)-4(1)+10(5)\\
                &= 46
            \end{align*}
            \begin{itemize}
                \item[$\therefore$] $T(2-x+3x^2) = 46$
            \end{itemize}
        \end{enumerate}

        \item \textbf{Exercise 7.1.4}\\In each case, find a linear transformation with the given properties and compute $T(v)$.
        \begin{enumerate}
            \setcounter{enumii}{2}
            \item $T : \textbf{P}_2 \to \textbf{P}_3 \mbox{; } T(x^2)=x^3 \mbox{, } T(x+1)=0 \mbox{, } T(x-1)=x \mbox{; } v=x^2+x+1$
            
            Diketahui:
            \begin{itemize}
                \item[] $T : \textbf{P}_2 \to \textbf{P}_3 $\item[]$ T(x^2)=x^3 $\item[]$ T(x+1)=0 $\item[]$ T(x-1)=x $\item[]$ v=x^2+x+1$
            \end{itemize}

            Ditanya:
            \begin{itemize}
                \item Nilai $T(v)$?
            \end{itemize}

            Jawab:
            \begin{align*}
                T(v) &= T(x^2+x+1)\\
                &= T(x^2) + T(x+1)\\
                &= x^3 + 0\\
                &= x^3
            \end{align*}
            \begin{itemize}
                \item[$\therefore$] $T(v)=x^3$
            \end{itemize}
        \end{enumerate}

        \item \textbf{Exercise 7.2.1}\\For each matrix $A$, find a basis for the kernel and image $T_{A}$, and find the rank and nullity of $T_{A}$.
        \begin{enumerate}
            \item $\begin{bmatrix}
                1 & 2 & -1 & 2\\
                3 & 1 & 0 & 2 \\
                1 & -3 & 2 & 0
            \end{bmatrix}$

            Diketahui:
            \begin{itemize}
                \item[] $A=\begin{bmatrix}
                    1 & 2 & -1 & 2\\
                    3 & 1 & 0 & 2 \\
                    1 & -3 & 2 & 0
                \end{bmatrix}$
            \end{itemize}

            Ditanya:
            \begin{itemize}
                \item Basis dari Ker($T_A$) dan Im($T_A$)?
                \item Rank dan nullity dari $T_A$?
            \end{itemize}
            
            Jawab:
            \begin{align*}
                A^T=\begin{bmatrix}
                    1&3&1\\2&1&-3\\-1&0&2\\1&2&0
                \end{bmatrix} R_2\to R_2-2R_1 &\begin{bmatrix}
                    1&3&1\\0&-5&-5\\-1&0&2\\1&2&0
                \end{bmatrix} \\R_3\to R_3 + R_1 &\begin{bmatrix}
                    1&3&1\\0&-5&-5\\0&3&3\\1&2&0
                \end{bmatrix} \\R_4\to R_4-R_1 &\begin{bmatrix}
                    1&3&1\\0&-5&-5\\0&3&3\\0&-1&-1
                \end{bmatrix} \\R_2\leftrightarrow R_4 &\begin{bmatrix}
                    1&3&1\\0&-1&-1\\0&3&3\\0&-5&-5
                \end{bmatrix} \\R_3\to R_3+3R_2 &\begin{bmatrix}
                    1&3&1\\0&-1&-1\\0&0&0\\0&-5&-5
                \end{bmatrix} \\R_4\to R_4-5R_2 &\begin{bmatrix}
                    1&3&1\\0&-1&-1\\0&0&0\\0&0&0
                \end{bmatrix}
            \end{align*}

            \begin{itemize}
                \item[$\therefore$] $\mbox{Basis dari Im(}T_A \mbox{) adalah} \left\{ \begin{bmatrix}
                    1\\3\\1
                \end{bmatrix}\mbox{, } \begin{bmatrix}
                    0\\-1\\-1
                \end{bmatrix} \right\}$ dan rank $T_A$ adalah 2. Rank didapat dari nilai dim(Im($T_A$)). 
            \end{itemize}
            \begin{align*}
                A = \begin{bmatrix}
                    1&2&-1&1\\3&1&0&2\\1&-3&2&0
                \end{bmatrix} R_2\to R_2-3B_1 &\begin{bmatrix}
                    1&2&-1&1\\0&-5&3&-1\\1&-3&2&0
                \end{bmatrix} \\R_3\to R_3-R_1 &\begin{bmatrix}
                    1&2&-1&1\\0&-5&3&-1\\0&-5&3&-1
                \end{bmatrix} \\R_3\to R_3-R_2 &\begin{bmatrix}
                    1&2&-1&1\\0&-5&3&-1\\0&0&0&0
                \end{bmatrix} \\R_2\to -\frac{R_2}{5} &\begin{bmatrix}
                    1&2&5&-1\\0&1&-\frac{3}{5}&\frac{1}{5}\\0&0&0&0
                \end{bmatrix} \\R_1\to R_1-2R_2 &\begin{bmatrix}
                    1&0&\frac{1}{5}&\frac{3}{5}\\0&1&-\frac{3}{5}&\frac{1}{5}\\0&0&0&0
                \end{bmatrix}
            \end{align*}

            \begin{itemize}
                \item[] Untuk mencari kernel, kita harus menyelesaikan:
            \end{itemize}
            \begin{align*}
                \begin{bmatrix}
                    1&0&\frac{1}{5}&\frac{3}{5}\\0&1&-\frac{3}{5}&\frac{1}{5}\\0&0&0&0
                \end{bmatrix} \begin{bmatrix}
                    x_1\\x_2\\x_3\\x_4
                \end{bmatrix} &= \begin{bmatrix}
                    0\\0\\0
                \end{bmatrix}
            \end{align*}

            \begin{itemize}
                \item[] Misal $x_3 = t, x_4 = s, \mbox{ maka } x_1=-\frac{3s}{5}-\frac{t}{5}, x_2=-\frac{s}{5}+\frac{3t}{5}$, didapat:
            \end{itemize}
            \begin{align*}
                \begin{bmatrix}
                    x_1\\x_2\\x_3\\x_4
                \end{bmatrix} = \begin{bmatrix}
                    -\frac{3s}{5}-\frac{t}{5}\\-\frac{s}{5}+\frac{3t}{5}\\t\\s
                \end{bmatrix} = \begin{bmatrix}
                    -\frac{1}{5}\\\frac{3}{5}\\1\\0
                \end{bmatrix}t + \begin{bmatrix}
                    -\frac{3}{5}\\-\frac{1}{5}\\0\\1
                \end{bmatrix}s
            \end{align*}

            \begin{itemize}
                \item[$\therefore$] $\mbox{Basis dari Ker(}T_A \mbox{) adalah} \left\{ \begin{bmatrix}
                    -\frac{1}{5}\\\frac{3}{5}\\1\\0
                \end{bmatrix}\mbox{, } \begin{bmatrix}
                    -\frac{3}{5}\\-\frac{1}{5}\\0\\1
                \end{bmatrix} \right\}$ dan nullity $T_A$ adalah 2. Nullity didapat dari nilai dim(Ker($T_A$)). 
            \end{itemize}

            \pagebreak

            \setcounter{enumii}{2}
            \item $\begin{bmatrix}
                1 & 2 & -1\\
                3 & 1 & 2\\
                4 & -1 & 5\\
                0 & 2 & -2
            \end{bmatrix}$

            Diketahui:
            \begin{itemize}
                \item[] $A=\begin{bmatrix}
                    1 & 2 & -1\\
                    3 & 1 & 2\\
                    4 & -1 & 5\\
                    0 & 2 & -2
                \end{bmatrix}$
            \end{itemize}

            Ditanya:
            \begin{itemize}
                \item Basis dari Ker($T_A$) dan Im($T_A$)?
                \item Rank dan nullity dari $T_A$?
            \end{itemize}
            
            Jawab:
            \begin{align*}
                A^T=\begin{bmatrix}
                    1&3&4&0\\2&1&-1&2\\-1&2&5&-2
                \end{bmatrix} R_2\to R_2-2R_1 &\begin{bmatrix}
                    1&3&4&0\\0&-5&-9&2\\-1&2&5&-2
                \end{bmatrix} \\R_3\to R_3 + R_1 &\begin{bmatrix}
                    1&3&4&0\\0&-5&-9&2\\0&5&9&-2
                \end{bmatrix} \\R_3\to R_3+R_2 &\begin{bmatrix}
                    1&3&4&0\\0&-5&-9&2\\0&0&0&0
                \end{bmatrix}
            \end{align*}

            \begin{itemize}
                \item[$\therefore$] $\mbox{Basis dari Im(}T_A \mbox{) adalah} \left\{ \begin{bmatrix}
                    1\\3\\4\\0
                \end{bmatrix}\mbox{, } \begin{bmatrix}
                    0\\-5\\-9\\2
                \end{bmatrix} \right\}$ dan rank $T_A$ adalah 2. Rank didapat dari nilai dim(Im($T_A$)). 
            \end{itemize}
            \begin{align*}
                A=\begin{bmatrix}
                    1 & 2 & -1\\
                    3 & 1 & 2\\
                    4 & -1 & 5\\
                    0 & 2 & -2
                \end{bmatrix} R_2\to R_2-3R_1 &\begin{bmatrix}
                    1 & 2 & -1\\
                    0 & -5 & 5\\
                    4 & -1 & 5\\
                    0 & 2 & -2
                \end{bmatrix} \\R_3\to R_3 - 4R_1 &\begin{bmatrix}
                    1 & 2 & -1\\
                    0 & -5 & 5\\
                    0 & -9 & 9\\
                    0 & 2 & -2
                \end{bmatrix} \\R_2\to -\frac{R_2}{5} &\begin{bmatrix}
                    1 & 2 & -1\\
                    0 & 1 & -1\\
                    0 & -9 & 9\\
                    0 & 2 & -2
                \end{bmatrix} \\R_3\to R_3 -9R_2 &\begin{bmatrix}
                    1 & 2 & -1\\
                    0 & 1 & -1\\
                    0 & 0 & 0\\
                    0 & 2 & -2
                \end{bmatrix} \\R_4\to R_4-2R_2 &\begin{bmatrix}
                    1 & 2 & -1\\
                    0 & 1 & -1\\
                    0 & 0 & 0\\
                    0 & 0 & 0
                \end{bmatrix} \\R_1\to R_1-2R_2 &\begin{bmatrix}
                    1 & 0 & 1\\
                    0 & 1 & -1\\
                    0 & 0 & 0\\
                    0 & 0 & 0
                \end{bmatrix}
            \end{align*}

            \begin{itemize}
                \item[] Untuk mencari kernel, kita harus menyelesaikan:
            \end{itemize}
            \begin{align*}
                \begin{bmatrix}
                    1 & 0 & 1\\
                    0 & 1 & -1\\
                    0 & 0 & 0\\
                    0 & 0 & 0
                \end{bmatrix} \begin{bmatrix}
                    x_1\\x_2\\x_3
                \end{bmatrix} &= \begin{bmatrix}
                    0\\0\\0
                \end{bmatrix}
            \end{align*}

            \begin{itemize}
                \item[] Misal $x_3 = t, \mbox{ maka } x_1=-t, x_2=t$, didapat:
            \end{itemize}
            \begin{align*}
                \begin{bmatrix}
                    x_1\\x_2\\x_3
                \end{bmatrix} = \begin{bmatrix}
                    -t\\t\\t
                \end{bmatrix} = \begin{bmatrix}
                    -1\\1\\1
                \end{bmatrix}t
            \end{align*}

            \begin{itemize}
                \item[$\therefore$] $\mbox{Basis dari Ker(}T_A \mbox{) adalah} \left\{ \begin{bmatrix}
                    -1\\1\\1
                \end{bmatrix} \right\}$ dan nullity $T_A$ adalah 1. Nullity didapat dari nilai dim(Ker($T_A$)). 
            \end{itemize}            
            
        \end{enumerate}

        \item \textbf{Exercise 7.2.2}\\In each case, (i) find a basis of Ker($T$), and (ii) find a basis of Im($T$). You may assume that $T$ is linear.
        \begin{enumerate}
            \setcounter{enumii}{5}
            \item $T : \textbf{M}_{22} \to \mathbb{R} \mbox{; } T \begin{bmatrix}
                a & b\\
                c & d
            \end{bmatrix} = a + d$

            Diketahui:
            \begin{itemize}
                \item[] $T : \textbf{M}_{22} \to \mathbb{R} $\item[]$ T \begin{bmatrix}
                a & b\\
                c & d
                \end{bmatrix} = a + d$\item[]$T$ linear
            \end{itemize}

            Ditanya:
            \begin{itemize}
                \item Basis dari Ker($T$) dan Im($T$)?
            \end{itemize}
            
            Jawab:
            \begin{align*}
                \mbox{Ker($T$)} &= \left\{ \left. \begin{bmatrix}
                    a&b\\c&d
                \end{bmatrix} \in \textbf{M}_{22} \right| a=-d \right\}
                \\
                &= \left\{\left.\begin{bmatrix}
                    -d&b\\c&d
                \end{bmatrix}\right|b, c, d \in \mathbb{R}\right\}
                \\
                &= \left\{\left. d \begin{bmatrix}
                    -1&0\\0&1
                \end{bmatrix}+b\begin{bmatrix}
                    0&1\\0&0
                \end{bmatrix}+c\begin{bmatrix}
                    0&0\\1&0
                \end{bmatrix}\right|b, c, d \in \mathbb{R}\right\}
                \\
                &= span\left\{\begin{bmatrix}
                    -1&0\\0&1
                \end{bmatrix}\mbox{, }\begin{bmatrix}
                    0&1\\0&0
                \end{bmatrix}\mbox{, }\begin{bmatrix}
                    0&0\\1&0
                \end{bmatrix}\right\}
            \end{align*}
            \begin{itemize}
                \item[] $\mbox{Terlihat set}\left\{\begin{bmatrix}
                    -1&0\\0&1
                \end{bmatrix}\mbox{, }\begin{bmatrix}
                    0&1\\0&0
                \end{bmatrix}\mbox{, }\begin{bmatrix}
                    0&0\\1&0
                \end{bmatrix}\right\} \mbox{ialah linearly independent.}$
            \end{itemize}

            \begin{itemize}
                \item[$\therefore$] $\left\{\begin{bmatrix}
                    -1&0\\0&1
                \end{bmatrix}\mbox{, }\begin{bmatrix}
                    0&1\\0&0
                \end{bmatrix}\mbox{, }\begin{bmatrix}
                    0&0\\1&0
                \end{bmatrix}\right\}$ adalah basis dari Ker($T$).
            \end{itemize}

            \begin{itemize}
                \item[] Untuk setiap $a \in \mathbb{R}$, kita mempunyai:
            \end{itemize}
            \begin{align*}
                \begin{bmatrix}
                    a&0\\0&0
                \end{bmatrix} \in \textbf{M}_{22} \mbox{ dan } T\left(\begin{bmatrix}
                    a&0\\0&0
                \end{bmatrix}\right) = a
            \end{align*}
            \begin{itemize}
                \item[] Oleh karena itu, $T$ surjektif dan mengimplikasi bahwa Im($T$) = $\mathbb{R}$.
            \end{itemize}

            \begin{itemize}
                \item[$\therefore$] $\left\{1\right\}$ adalah basis dari Im($T$).
            \end{itemize}
            
        \end{enumerate}

        \pagebreak

        \item \textbf{Exercise 7.2.14}\\Consider $V = \left\{ \left. \begin{bmatrix}
            a & b\\
            c & d
        \end{bmatrix} \right| a+c=b+d \right\}$
        \begin{enumerate}
            \item Consider $S : \textbf{M}_{22} \to \mathbb{R} \mbox{ with } S \begin{bmatrix}
                a & b\\
                c & d
            \end{bmatrix} = a+c-b-d$. Show that $S$ is linear and onto and that $V$ is a subspace of \textbf{M}$_{22}$. Compute dim($V$).
            
            Diketahui:
            \begin{itemize}
                \item[] $V = \left\{ \left. \begin{bmatrix}
                    a & b\\
                    c & d
                \end{bmatrix} \right| a+c=b+d \right\}$\item[]$S : \textbf{M}_{22} \to \mathbb{R}$ \item[] $S \begin{bmatrix}
                    a & b\\
                    c & d
                \end{bmatrix} = a+c-b-d$
            \end{itemize}

            Ditanya:
            \begin{itemize}
                \item Buktikan $S$ linear dan onto, serta $V$ adalah subspace dari $\textbf{M}_{22}$.
                \item Nilai dim($V$)?
            \end{itemize}
            
            Jawab:
            \begin{align}
                S(\textbf{X}_1 + \textbf{X}_2) &=S\left\{\begin{bmatrix}
                    a_1&b_1\\c_1&d_1
                \end{bmatrix}+\begin{bmatrix}
                    a_2&b_2\\c_2&d_2
                \end{bmatrix}\right\}\notag\\ &=S\begin{bmatrix}
                    a_1+a_2&b_1+b_2\\c_1+c_2&d_1+d_2
                \end{bmatrix}\notag\\ &=(a_1+a_2)+(c_1+c_2)-(b_1+b_2)_(d_1+d_2)
                \notag\\ &= (a_1+c_1-b_1-d_1)+(a_2+c_2-b_2-d_2)
                \notag\\ &= S\begin{bmatrix}
                    a_1&b_1\\c_1&d_1
                \end{bmatrix}+S\begin{bmatrix}
                    a_2&b_2\\c_2&d_2
                \end{bmatrix} \notag\\ &= S(\textbf{X}_1)+S(\textbf{X}_2)
            \end{align}
            \begin{align}
                S(r\textbf{X}_1) &=S\left(r\begin{bmatrix}
                    a_1&b_1\\c_1&d_1
                \end{bmatrix}\right) \notag\\ &=S\begin{bmatrix}
                    ra_1&rb_1\\rc_1&rd_1
                \end{bmatrix} = ra_1+rc_1-rb_1-rd_1 \notag\\ &=r(a_1+c_1-b_1-d_1) =rS(\textbf{X}_1)
            \end{align}
            \begin{itemize}
                \item[$\therefore$] $S$ adalah linear berdasarkan persamaan (1) dan (2). $\blacksquare$
            \end{itemize}

            \begin{itemize}
                \item[] Misal $r\in \mathbb{R}$. Kita mendapatkan: 
                \begin{align*}
                    S\begin{bmatrix}
                        r&0\\0&0
                    \end{bmatrix}=r
                \end{align*}
            \end{itemize}
            \begin{itemize}
                \item[$\therefore$] $S$ adalah onto. $\blacksquare$
            \end{itemize}

            \begin{align}
                V &= \left\{ \left. \begin{bmatrix}
                    a & b\\
                    c & d
                \end{bmatrix} \right| a+c=b+d \right\} \notag\\
                0+0=0+0 &\Rightarrow \begin{bmatrix}
                    0&0\\0&0
                \end{bmatrix} \in \mathbb{R} \notag\\ &\Rightarrow  0\in \mathbb{R}  
            \end{align}
            \begin{align}
                \textbf{X}_1, \textbf{X}_2 \in V &\Rightarrow \begin{bmatrix}
                    a_1&b_1\\c_1&d_1
                \end{bmatrix} \in V \land \begin{bmatrix}
                    a_2&b_2\\c_2&d_2
                \end{bmatrix} \in V, a_i+c_i=b_i+d_i, i=1, 2 \notag
                \\ &\Rightarrow \begin{bmatrix}
                    a_1+a_2&b_1+b_2\\c_1+c_2&d_1+d_2
                \end{bmatrix} \in V, \sum_{i=1}^{2} (a_i+c_i) = \sum_{i=1}^{2} (b_i+d_i) \notag
                \\ &\Rightarrow \begin{bmatrix}
                    a_1&b_1\\c_1&d_1
                \end{bmatrix} + \begin{bmatrix}
                    a_2&b_2\\c_2&d_2
                \end{bmatrix} \in V, \sum_{i=1}^{2} (a_i+c_i) = \sum_{i=1}^{2} (b_i+d_i) \notag
                \\ &\Rightarrow \textbf{X}_1 + \textbf{X}_2 \in \mathbb{R}
            \end{align}
            \begin{align}
                \textbf{X}_1 \in V, r \in \mathbb{R} &\Rightarrow \begin{bmatrix}
                    a_1&b_1\\c_1&d_1
                \end{bmatrix} \in \mathbb{R}, a_1 + c_1 = b_1 + d_1, a \in \mathbb{R} \notag
                \\ &\Rightarrow \begin{bmatrix}
                    ra_1&rb_1\\rc_1&rd_1
                \end{bmatrix} \in V, ra_1 + rc_1 = rb_1 + rd_1, r \in \mathbb{R} \notag
                \\ &\Rightarrow r\textbf{X}_1 \in V, r \in \mathbb{R}
            \end{align}

            \begin{itemize}
                \item[$\therefore$] $V$ adalah subspace dari $\textbf{M}_{22}$ berdasarkan persamaan (3), (4), dan (5). $\blacksquare$
            \end{itemize}

            \begin{align*}
                \begin{bmatrix}
                    a&b\\c&d
                \end{bmatrix} \in V &\Rightarrow a+c = b+d 
                \\ &\Leftrightarrow a = b+d-c
                \\ &\Leftrightarrow \begin{bmatrix}
                    b+d-c&b\\c&d
                \end{bmatrix} \in V
                \\ &\Rightarrow \left\{ b\begin{bmatrix}
                    1&1\\0&0
                \end{bmatrix}+c\begin{bmatrix}
                    -1&0\\1&0
                \end{bmatrix}+d\begin{bmatrix}
                    1&0\\0&1
                \end{bmatrix}; b, c, d \in \mathbb{R} \right\} = V
                \\ &\Rightarrow span \left\{\begin{bmatrix}
                    1&1\\0&0
                \end{bmatrix}, \begin{bmatrix}
                    -1&0\\1&0
                \end{bmatrix}, \begin{bmatrix}
                    1&0\\0&1
                \end{bmatrix} \right\} = V
            \end{align*}

            \begin{itemize}
                \item[] $\mbox{Terlihat set}\left\{\begin{bmatrix}
                    1&1\\0&0
                \end{bmatrix}, \begin{bmatrix}
                    -1&0\\1&0
                \end{bmatrix}, \begin{bmatrix}
                    1&0\\0&1
                \end{bmatrix} \right\} \mbox{ialah linearly independent.}$
            \end{itemize}
            \begin{itemize}
                \item[$\therefore$] Karena set tersebut merupakan basis dari subset $V$, dim($V$)$=3$.
            \end{itemize}

            \pagebreak

            \item Consider $T : V \to \mathbb{R} \mbox{ with } T \begin{bmatrix}
                a & b\\
                c & d
            \end{bmatrix} = a+c$. Show that $T$ is linear and onto, and use this information to compute dim(Ker($T$)).

            Diketahui:
            \begin{itemize}
                \item[] $V = \left\{ \left. \begin{bmatrix}
                    a & b\\
                    c & d
                \end{bmatrix} \right| a+c=b+d \right\}$\item[]$T : V \to \mathbb{R}$ \item[] $T \begin{bmatrix}
                    a & b\\
                    c & d
                \end{bmatrix} = a+c$
            \end{itemize}

            Ditanya:
            \begin{itemize}
                \item Buktikan $T$ linear dan onto.
                \item Nilai dim(Ker($T$))?
            \end{itemize}
            
            Jawab:
            \setcounter{equation}{0}
            \begin{align}
                T(\textbf{X}_1 + \textbf{X}_2) &=T\left\{\begin{bmatrix}
                    a_1&b_1\\c_1&d_1
                \end{bmatrix}+\begin{bmatrix}
                    a_2&b_2\\c_2&d_2
                \end{bmatrix}\right\}\notag\\ &=T\begin{bmatrix}
                    a_1+a_2&b_1+b_2\\c_1+c_2&d_1+d_2
                \end{bmatrix}\notag\\ &=(a_1+a_2)+(c_1+c_2)
                \notag\\ &= (a_1+c_1)+(a_2+c_2)
                \notag\\ &= T\begin{bmatrix}
                    a_1&b_1\\c_1&d_1
                \end{bmatrix}+T\begin{bmatrix}
                    a_2&b_2\\c_2&d_2
                \end{bmatrix} \notag\\ &= T(\textbf{X}_1)+T(\textbf{X}_2)
            \end{align}
            \begin{align}
                T(r\textbf{X}_1) &=T\left(r\begin{bmatrix}
                    a_1&b_1\\c_1&d_1
                \end{bmatrix}\right) \notag\\ &=T\begin{bmatrix}
                    ra_1&rb_1\\rc_1&rd_1
                \end{bmatrix} = ra_1+rc_1 \notag\\ &=r(a_1+c_1) =rS(\textbf{X}_1)
            \end{align}
            \begin{itemize}
                \item[$\therefore$] $T$ adalah linear berdasarkan persamaan (1) dan (2). $\blacksquare$
            \end{itemize}

            \begin{itemize}
                \item[] Misal $r\in \mathbb{R}$. Kita mendapatkan: 
                \begin{align*}
                    T\begin{bmatrix}
                        r&0\\0&0
                    \end{bmatrix}=r
                \end{align*}
            \end{itemize}
            \begin{itemize}
                \item[$\therefore$] $T:V\to \mathbb{R}$ adalah onto. $\blacksquare$
            \end{itemize}

            \begin{itemize}
                \item[] Dengan demikian, kita mendapatkan $T(V) = \mathbb{R}$, dan menghasilkan nilai dim(Im($T$)) $=1$
            \end{itemize}
            \begin{align*}
                \mbox{dim($V$)}&=\mbox{dim(Im($T$))}+\mbox{dim(Ker($T$))}
                \\\Rightarrow \mbox{dim(Ker($T$))} &= \mbox{dim($V$)}-\mbox{dim(Im($T$))}
                \\\Rightarrow \mbox{dim(Ker($T$))} &= 3 - 1 = 2
            \end{align*}

            \begin{itemize}
                \item[$\therefore$] Nilai dim(Ker($T$)) adalah 2.
            \end{itemize}

        \end{enumerate}
    \end{enumerate}

\section*{Orthogonality}
    \subsection*{Orthogonal Complements and Projections}
        \begin{enumerate}
            \item \textbf{Exercise 8.1.1}\\In each case, use the Gram-Schmidt algorithm to convert the given basis $B$ of $V$ into an orthogonal basis.
            \begin{enumerate}
                \setcounter{enumii}{2}
                \item $V = \mathbb{R}^3 \mbox{, } B = \{(1, -1, 1), \mbox{ }(1, 0, 1), \mbox{ }(1, 1, 2)\}$
                
                Diketahui:
                \begin{itemize}
                    \item[] $V=\mathbb{R}^3$
                    \item[] $B={(1, -1, 1), (1, 0, 1), (1, 1, 2)}$
                    \item[] $\textbf{x}_1=(1, -1, 1)$
                    \item[] $\textbf{x}_2=(1, 0, 1)$
                    \item[] $\textbf{x}_3=(1, 1, 2)$
                \end{itemize}

                Ditanya:
                \begin{itemize}
                    \item Orthogonal basis dari basis $B$ of $V$?
                \end{itemize}
            
                Jawab:
                \setcounter{equation}{0}
                \begin{align}
                    \notag \textbf{f}_1 &= \textbf{x}_1
                    \\ &= (1, -1, 1)
                \end{align}
                \begin{align}
                    \notag \textbf{f}_2 &= \textbf{x}_2 - \frac{\textbf{x}_2 \cdot \textbf{f}_1}{\Vert \textbf{f}_1 \Vert^2} \textbf{f}_1
                    \\\notag &= (1,0,1) - \frac{(1, 0, 1) (1, -1, 1)}{\Vert(1, -1, 1)\Vert^2} (1, -1, 1)
                    \\\notag &= (1,0,1) - \frac{2}{3} (1, -1, 1)
                    \\ &= (\frac{1}{3}, \frac{2}{3}, \frac{1}{3}) = \frac{1}{3}(1, 2, 1)
                \end{align}
                \begin{align}
                    \notag \textbf{f}_3 &= \textbf{x}_3 - \frac{\textbf{x}_3 \cdot \textbf{f}_1}{\Vert\textbf{f}_1\Vert^2}\textbf{f}_1 - \frac{\textbf{x}_3 \cdot \textbf{f}_2}{\Vert\textbf{f}_2\Vert^2}\textbf{f}_2
                    \\\notag &= (1, 1, 2) - \frac{(1, 1, 2)(1,-1,1)}{\Vert(1, -1, 1)\Vert^2}(1,-1,1) - \frac{(1,1,2)\cdot\frac{1}{3}(1,2,1)}{\Vert\frac{1}{3}(1,2,1)\Vert^2}\cdot\frac{1}{3}(1,2,1)
                    \\\notag &= (1, 1, 2) - \frac{2}{3}(1, -1, 1) - \frac{5}{3}(1,2,1)
                    \\ &= (-\frac{4}{3}, -\frac{5}{3}, -\frac{1}{3}) = \frac{1}{3}(-4,-5,-1)
                \end{align}

                \begin{itemize}
                    \item[$\therefore$] Orthogonal basisnya adalah $\{(1, -1, 1), \frac{1}{3}(1, 2, 1), \frac{1}{3}(-4, -5, -1)\}$.
                \end{itemize}

            \end{enumerate}
            
            \item \textbf{Exercise 8.1.2}\\In each case, write \textbf{x} as the sum of a vector in $U$ and a vector in $U^\perp$.
            \begin{enumerate}
                \setcounter{enumii}{5}
                \item $\textbf{x} = (a, b, c, d),\\U = \mbox{ span } \{(1, -1, 2, 0), \mbox{ }(-1, 1, 1, 1)\}$
                
                Diketahui:
                \begin{itemize}
                    \item[] $\textbf{x}=(a,b,c,d)$
                    \item[] $U=\mbox{span}{(1,-1,2,0),(-1,1,1,1)}$
                \end{itemize}

                Ditanya:
                \begin{itemize}
                    \item Tulis \textbf{x} sebagai sum dari vector in $U$ dan vector in $U^\perp$.
                \end{itemize}
            
                Jawab:
                \begin{itemize}
                    \item[] Misal $\textbf{e}_1 = (1,-1,2,0)$ dan $\textbf{e}_2 = (-1,1,1,1)$ maka:
                \end{itemize}
                \begin{align*}
                    \textbf{x}_1 &= \mbox{proj}_U \textbf{x}
                    \\ &= \frac{a-b+2c}{6}(1,-1,2,0)+\frac{-a+b+c+d}{4}(-1,1,1,1)
                    \\ &= (\frac{5a-5b+c-3d}{12}, \frac{-5a+5b-c+3d}{12},
                    \\ &\frac{a-b+11c+3d}{12}, \frac{-3a+3b+3c+3d}{12})
                    \\ \textbf{x}_2 &= \textbf{x} - \textbf{x}_1
                    \\ &= (\frac{7a+5b-c+3d}{12}, \frac{5a+7b+c-3d}{12}
                    \\ &\frac{-a-b+c-3d}{12}, \frac{3a-3b-3c+9d}{12})
                \end{align*}

                \begin{itemize}
                    \item[$\therefore$] $\textbf{x}=\textbf{x}_1 + \textbf{x}_2$
                \end{itemize}
            \end{enumerate}

            \pagebreak
            
            \item \textbf{Exercise 8.1.3}\\Let $\textbf{x}=(1, -2, 1, 6)$ in $\mathbb{R}^4$, and let $U = \mbox{ span } \{(2, 1, 3, -4), \mbox{ }(1, 2, 0, 1)\}$.
            \begin{enumerate}
                \item Compute proj$_U$ \textbf{x}.
                
                \item Show that $\{(1, 0, 2, -3), \mbox{ }(4, 7, 1, 2)\}$ is another orthogonal basis of $U$.
                
                \item Use the basis in part (b) to compute proj$_U$ \textbf{x}.
                
                Diketahui:
                \begin{itemize}
                    \item[] $\textbf{x}=(1, -2, 1, 6) \in \mathbb{R}^4$
                    \item[] $U = \mbox{ span } \{(2, 1, 3, -4), \mbox{ }(1, 2, 0, 1)\}$
                \end{itemize}

                Ditanya:
                \begin{itemize}
                    \item Nilai $\mbox{proj}_U \textbf{x}$?
                    \item Tunjukkan bahwa $\{(1,0,2,-3), (4,7,1,2)\}$ adalah contoh lain orthogonal basis $U$.
                    \item Gunakan basis di atas untuk menghitung nilai $\mbox{proj}_U \textbf{x}$. 
                \end{itemize}
            
                Jawab:
                \begin{align*}
                    \mbox{proj}_U \textbf{x} &= \frac{\textbf{x} \cdot \textbf{f}_1}{\Vert \textbf{f}_1 \Vert^2} \textbf{f}_1 + \frac{\textbf{x} \cdot \textbf{f}_2}{\Vert \textbf{f}_2 \Vert^2} \textbf{f}_2
                    \\ &= \frac{(1,-2,1,6) (2,1,3,-4)}{\Vert (2,1,3,-4) \Vert^2} (2,1,3,-4) 
                    \\ &+ \frac{(1,-2,1,6) (1,2,0,1)}{\Vert (1,2,0,1) \Vert^2} (1,2,0,1)
                    \\ &= -\frac{21}{30} (2,1,3,-4)+ \frac{3}{6}(1,2,0,1)
                    \\ &= \frac{3}{10}(-3,1,-7,11)
                \end{align*}
                \begin{itemize}
                    \item[$\therefore$] $\mbox{proj}_U \textbf{x} = \frac{3}{10}(-3,1,-7,11)$.
                \end{itemize}
                \begin{align*}
                    U &= \mbox{span}\{(2,1,3,-4),(1,2,0,1)\}
                    \\ &= \alpha(2,1,3,-4) + \beta(1,2,0,1)
                    \\ &= (2\alpha+\beta,\alpha+2\beta,3\alpha,-4\alpha+\beta)
                    \\\mbox{f}_1 &= (1,0,2,-3) ; \alpha = \frac{2}{3}, \beta = -\frac{1}{3}
                    \\\mbox{f}_2 &= (4,7,1,2) ; \alpha = \frac{1}{3}, \beta = \frac{10}{3}
                \end{align*}
                \begin{align*}
                    \mbox{f}_1\cdot\mbox{f}_2 = 4+0+2-6 = 0
                \end{align*}
                \begin{itemize}
                    \item[$\therefore$] $\{(1,0,2,-3),(4,7,1,2)\}$ juga $\mbox{span } \{(2, 1, 3, -4), \mbox{ }(1, 2, 0, 1)\}$
                \end{itemize}
                \begin{align*}
                    \mbox{proj}_U \textbf{x} &= \frac{\textbf{x} \cdot \textbf{f}_1}{\Vert \textbf{f}_1 \Vert^2} \textbf{f}_1 + \frac{\textbf{x} \cdot \textbf{f}_2}{\Vert \textbf{f}_2 \Vert^2} \textbf{f}_2
                    \\ &= \frac{(1,-2,1,6) (1,0,2,-3)}{\Vert (1,0,2,-3) \Vert^2} (1,0,2,-3) 
                    \\ &+ \frac{(1,-2,1,6) (4,7,1,2)}{\Vert (4,7,1,2) \Vert^2} (4,7,1,2)
                    \\ &= -\frac{15}{14} (1,0,2,-3)+ \frac{3}{70}(4,7,1,2)
                    \\ &= \frac{3}{10}(-3,1,-7,11)
                \end{align*}
                \begin{itemize}
                    \item[$\therefore$] Nilai $\mbox{proj}_U \textbf{x}$ tetap sama walaupun menggunakan basis yang berbeda.
                \end{itemize}
                
            \end{enumerate}
            
            \item \textbf{Exercise 8.1.4}\\In each case, use the Gram-Schmidt algorithm to find an orthogonal basis of the subspace $U$, and find the vector in $U$ closest to \textbf{x}.
            \begin{enumerate}
                \setcounter{enumii}{2}
                \item $U = \mbox{ span } \{(1, 0, 1, 0), \mbox{ }(1, 1, 1, 0), \mbox{ }(1, 1, 0, 0)\},\\\textbf{x}=(2, 0, -1, 3)$
                
                Diketahui:
                \begin{itemize}
                    \item[] $U=\mbox{span}\{(1,0,1,0), (1,1,1,0), (1,1,0,0)\}$
                    \item[] $\textbf{x} = (2,0,-1,3)$
                \end{itemize}

                Ditanya:
                \begin{itemize}
                    \item Orthogonal basis dari subspace $U$?
                    \item Vector dalam $U$ yang terdekat dengan \textbf{x}?
                \end{itemize}
            
                Jawab:
                \setcounter{equation}{0}
                \begin{align}
                    \notag \textbf{f}_1 &= \textbf{x}_1
                    \\ &= (1,0,1,0)
                \end{align}
                \begin{align}
                    \notag \textbf{f}_2 &= \textbf{x}_2 - \frac{\textbf{x}_2 \cdot \textbf{f}_1}{\Vert \textbf{f}_1 \Vert^2} \textbf{f}_1
                    \\\notag &= (1,1,1,0) - \frac{(1,1,1,0) (1,0,1,0)}{\Vert(1,0,1,0)\Vert^2} (1,0,1,0)
                    \\\notag &= (1,1,1,0) - \frac{2}{3} (1,0,1,0)
                    \\ &= (0,1,0,0)
                \end{align}
                \begin{align}
                    \notag \textbf{f}_3 &= \textbf{x}_3 - \frac{\textbf{x}_3 \cdot \textbf{f}_1}{\Vert\textbf{f}_1\Vert^2}\textbf{f}_1  - \frac{\textbf{x}_3 \cdot \textbf{f}_2}{\Vert\textbf{f}_2\Vert^2}\textbf{f}_2
                    \\\notag &= (1,1,0,0) - \frac{(1,1,0,0)(1,0,1,0)}{\Vert(1,0,1,0)\Vert^2}(1,0,1,0) 
                    \\\notag &- \frac{(1,1,0,0)(0,1,0,0)}{\Vert (0,1,0,0) \Vert^2}(0,1,0,0)
                    \\\notag &= (1,1,0,0) - \frac{1}{2}(1,0,1,0) - (0,1,0,0)
                    \\ &= \frac{1}{2}(1,0,-1,0)
                \end{align}
                \begin{itemize}
                    \item[$\therefore$] $\{(1,0,1,0),(0,1,0,0),(\frac{1}{2}(1,0,-1,0))\}$ adalah orthogonal basis dari subspace $U$.
                \end{itemize}

                \begin{align*}
                    \mbox{proj}_U &= \frac{(\textbf{x}) \cdot \textbf{f}_1}{\Vert\textbf{f}_1\Vert^2}\textbf{f}_1 + \frac{\textbf{x} \cdot \textbf{f}_2}{\Vert\textbf{f}_2\Vert^2}\textbf{f}_2 + \frac{\textbf{x} \cdot \textbf{f}_3}{\Vert\textbf{f}_3\Vert^2}\textbf{f}_3
                    \\ &= \frac{(2,0,-1,3) \cdot (1,0,1,0)}{\Vert(1,0,1,0)\Vert^2}(1,0,1,0) + \frac{(2,0,-1,3) \cdot (0,1,0,0)}{\Vert(0,1,0,0)\Vert^2}(0,1,0,0)
                    \\ &+ \frac{(2,0,-1,3) \cdot \frac{1}{2}(1,0,-1,0)}{\Vert\frac{1}{2}(1,0,-1,0)\Vert^2}\frac{1}{2}(1,0,-1,0)
                    \\ &= \frac{1}{2}(1,0,1,0) + 0 + \frac{3}{2}(1,0,-1,0)
                    \\ &= \frac{1}{2}(4,0,-2,0) = (2,0,-1,0)
                \end{align*}
                \begin{itemize}
                    \item[$\therefore$] $(2,0,-1,0)$ adalah vector dalam $U$ yang terdekat dengan \textbf{x}.
                \end{itemize}
            \end{enumerate}
            
            \item \textbf{Exercise 8.1.5}\\Let $U = \mbox{ span } \{\textbf{v}_1, \textbf{v}_2, \mbox{ }\dots\mbox{ }, \textbf{v}_k\} \mbox{; } \textbf{v}_i \mbox{ in } \mathbb{R}^n$, and let $A$ be the $k \times n$ matrix with the \textbf{v}$_i$ as rows.
            \begin{enumerate}
                \item Show that $U^\perp = \{\textbf{x} \mbox{ }|\mbox{ } \textbf{x} \in \mathbb{R}^n, A\textbf{x}^T=0\}$.
                \item Use part (a) to find $U^\perp \mbox{ if}\\U = \mbox{ span } \{(1, -1, 2, 1), \mbox{ }(1, 0, -1, 1)\}$.
                
                Diketahui:
                \begin{itemize}
                    \item[] $U = \mbox{ span } \{\textbf{v}_1, \textbf{v}_2, \mbox{ }\dots\mbox{ }, \textbf{v}_k\}$
                    \item[] $A$ $k \times n$ matriks dengan $\textbf{v}_i$ sebagai baris.
                \end{itemize}

                Ditanya:
                \begin{itemize}
                    \item Tunjukkan bahwa  $U^\perp = \{\textbf{x} \mbox{ }|\mbox{ } \textbf{x} \in \mathbb{R}^n, A\textbf{x}^T=0\}$.
                    \item Nilai $U^\perp \mbox{ jika }U = \mbox{ span } \{(1, -1, 2, 1), \mbox{ }(1, 0, -1, 1)\}$?
                \end{itemize}
            
                \pagebreak
            
                Jawab:
                \begin{itemize}
                    \item[] $U^\perp \mbox{ mempunyai semua element } \mbox{ seperti } v_1x=0, v_2x=0, \dots, v_kx=0$
                    \item[] $A \mbox{ adalah matriks dengan } v_i \mbox{ sebagai baris.}$ 
                \end{itemize}

                \begin{align*}
                    A = \begin{bmatrix}
                        v_1 \\ v_2 \\ \dots \\ v_k
                    \end{bmatrix} \textbf{x}^T
                \end{align*}

                \begin{itemize}
                    \item[] Maka:
                \end{itemize}

                \begin{align*}
                    A \textbf{x}^T = \begin{bmatrix}
                        v_1 x \\ v_2 x \\ \dots \\ v_k x
                    \end{bmatrix} = \begin{bmatrix}
                        0\\0\\\dots\\0
                    \end{bmatrix}
                \end{align*}

                \begin{itemize}
                    \item[$\therefore$] $A\textbf{x}^T = 0$.
                \end{itemize}

                \begin{itemize}
                    \item[] $U= \mbox{ span }\{(1,-1,2,1),(1,0,-1,1)\}$
                \end{itemize}

                \begin{align*}
                    &\begin{bmatrix}
                        1 & -1 & 2 & 1\\1 & 0 & -1 & 1
                    \end{bmatrix} \begin{bmatrix}
                        \textbf{x}_1 \\ \textbf{x}_2 \\ \textbf{x}_3 \\ \textbf{x}_4
                    \end{bmatrix} = \begin{bmatrix}
                        0 \\ 0
                    \end{bmatrix}
                    \\ &B_2 \to B_2 - B_1
                    \\ &\begin{bmatrix}
                        1 & -1 & 2 & 1 \\ 0 & 1 & -3 & 0
                    \end{bmatrix} \begin{bmatrix}
                        \textbf{x}_1 \\ \textbf{x}_2 \\ \textbf{x}_3 \\ \textbf{x}_4
                    \end{bmatrix} = \begin{bmatrix}
                        0 \\ 0
                    \end{bmatrix}
                \end{align*}

                \begin{align*}
                    \textbf{x}_2 - 3\textbf{x}_3 &= 0
                    \\ \textbf{x}_2 &= 3\textbf{x}_3
                    \\ \textbf{x}_1 - \textbf{x}_2 + 2\textbf{x}_3 + \textbf{x}_4 &= 0
                    \\ \textbf{x}_1 - 3\textbf{x}_3 + \textbf{4} &= 0
                    \\ \textbf{x}_1 &= \textbf{x}_3 - \textbf{x}_4
                \end{align*}

                \pagebreak

                \begin{itemize}
                    \item[] Maka:
                \end{itemize}
                \begin{align*}
                    \begin{bmatrix}
                        \textbf{x}_1 \\ \textbf{x}_2 \\ \textbf{x}_3 \\ \textbf{x}_4
                    \end{bmatrix} &= \begin{bmatrix}
                        \textbf{x}_3 - \textbf{x}_4 \\ 3\textbf{x}_3 \\ \textbf{x}_3 \\ \textbf{x}_4
                    \end{bmatrix}
                    \\ \begin{bmatrix}
                        \textbf{x}_1 \\ \textbf{x}_2 \\ \textbf{x}_3 \\ \textbf{x}_4
                    \end{bmatrix} &= \textbf{x}_3 \begin{bmatrix}
                        1 & 3 & 1 & 0
                    \end{bmatrix} + \textbf{x}_4 \begin{bmatrix}
                        -1 \\ 0 \\ 0 \\ 1
                    \end{bmatrix}
                \end{align*}

                \begin{itemize}
                    \item[$\therefore$] $U^\perp = \mbox{ span } \{(1,3,1,0), (-1,0,0,1)\}$.
                \end{itemize}

            \end{enumerate}
        \end{enumerate}

    \subsection*{Orthogonal Diagonalization}
        \begin{enumerate}
            \setcounter{enumi}{5}
            \item \textbf{Exercise 8.2.1}\\Normalize the rows to make each of the following matrices orthogonal.
            \begin{enumerate}
                \setcounter{enumii}{2}
                \item $A = \begin{bmatrix}
                    1 & 2\\
                    -4 & 2
                \end{bmatrix}$

                Diketahui:
                \begin{itemize}
                    \item[] $A = \begin{bmatrix}
                        1 & 2 \\ -4 & 2
                    \end{bmatrix}$
                \end{itemize}

                Ditanya:
                \begin{itemize}
                    \item Normalize $A$.
                \end{itemize}

                Jawab:
                \begin{align*}
                    \Vert (1, 2) \Vert &= \sqrt{1+4} = \sqrt{5}
                    \\ \Vert (-4, 2) \Vert &= \sqrt{16 + 4} = 2\sqrt{5}
                \end{align*}

                \begin{itemize}
                    \item[$\therefore$] $A \sim \begin{bmatrix}
                        \frac{\sqrt{5}}{5} & \frac{2\sqrt{5}}{5}
                        \\ \frac{-4\sqrt{5}}{5} & \frac{2\sqrt{5}}{5}
                    \end{bmatrix} = \begin{bmatrix}
                        \frac{\sqrt{5}}{5} & \frac{2\sqrt{5}}{5}
                        \\ \frac{-2\sqrt{5}}{5} & \frac{\sqrt{5}}{5}
                    \end{bmatrix}$
                \end{itemize}
            \end{enumerate}
            
            \item \textbf{Exercise 8.2.5}\\For each matrix $A$, find an orthogonal matrix $P$ such that $P^{-1}AP$ is diagonal.
            \begin{enumerate}
                \setcounter{enumii}{6}
                \item $A = \begin{bmatrix}
                    5&3&0&0\\
                    3&5&0&0\\
                    0&0&7&1\\
                    0&0&1&7
                \end{bmatrix}$

                Diketahui:
                \begin{itemize}
                    \item[] $A = \begin{bmatrix}
                        5 & 3 & 0 & 0
                        \\ 3 & 5 & 0 & 0
                        \\ 0 & 0 & 7 & 1
                        \\ 0 & 0 & 1 & 7
                    \end{bmatrix}$
                \end{itemize}

                Ditanya:
                \begin{itemize}
                    \item Orthogonal matriks $P$, di mana $P^{-1}AP$ adalah diagonal?
                \end{itemize}

                Jawab:
                \begin{itemize}
                    \item[] Mencari eigenvalues dan eigenvectors
                \end{itemize}
                \begin{align*}
                    \begin{bmatrix}
                        5-\lambda & 3 & 0 & 0
                        \\ \lambda^4 - 24\lambda^3 + 204\lambda^2 - 704\lambda = 0
                        \\ \lambda_1 = 6; \lambda_2 = 2; \lambda_3 = 8; \lambda_4 = 8
                    \end{bmatrix}
                \end{align*}

                \begin{itemize}
                    \item[] Untuk $\lambda_1 = 6$
                \end{itemize}

                \begin{align*}
                    \begin{bmatrix}
                        -1 & 3 & 0 & 0
                        \\ 3 & -1 & 0 & 0
                        \\ 0 & 0 & 1 & 1
                        \\ 0 & 0 & 1 & 1
                    \end{bmatrix}
                \end{align*}

                \begin{itemize}
                    \item[] Null space dari matriks adalah: $\begin{bmatrix}
                        0\\0\\-1\\1
                    \end{bmatrix}$
                    \item[] Untuk $\lambda_2 = 2$
                    \item[] Null space dari matriks adalah: $\begin{bmatrix}
                        -1&1&0&0
                    \end{bmatrix}$
                    \item[] Untuk $\lambda_3 = 8$
                    \item[] Null space dari matriks adalah: $\begin{bmatrix}
                        1&1&0&0
                    \end{bmatrix}$
                    \item[] Untuk $\lambda_4 = 8$
                    \item[] Null space dari matriks adalah: $\begin{bmatrix}
                        0&0&1&1
                    \end{bmatrix}$
                    \item[$\therefore$] $P$ matriks adalah: 
                \end{itemize}
                \begin{align*}
                    P = \begin{bmatrix}
                        0&-1&1&0
                        \\ 0&1&1&0
                        \\ -1&0&0&1
                        \\ 1&0&0&1
                    \end{bmatrix}
                \end{align*}
                \begin{itemize}
                    \item[] Dari diagonal matriks, $D$ adalah:
                \end{itemize}
                \begin{align*}
                    \lambda_1 = 6; \lambda_2; \lambda_3 = 8; \lambda_4 = 8
                    \\ P^{-1}AP=\begin{bmatrix}
                        6&0&0&0\\0&2&0&0\\0&0&8&0\\0&0&0&8
                    \end{bmatrix} = D
                \end{align*}

            \end{enumerate}
            
            \item \textbf{Exercise 8.2.7}\\Consider $A = \begin{bmatrix}
                0&0&a\\
                0&b&0\\
                a&0&0
            \end{bmatrix}$. Show that $c_A(x)=(x-b)(x-a)(x+a)$ and find an orthogonal matrix $P$ such that $P^{-1}AP$ is diagonal.

            Diketahui:
            \begin{itemize}
                \item[] $A = \begin{bmatrix}
                    0&0&a\\0&b&0\\a&0&0
                \end{bmatrix}$
            \end{itemize}

            Ditanya:
            \begin{itemize}
                \item Tunjukkan $c_A(x)=(x-b)(x-a)(x+a)$.
                \item Orthogonal matriks $P$, di mana $P^{-1}AP$ adalah diagonal?
            \end{itemize}
            
            Jawab:
            \begin{itemize}
                \item[] Mencari karakteristik polinomial, eigenvalues, dan eigenvectors
            \end{itemize}
            \begin{align*}
                \mbox{det}(\textbf{x}I-A) &= \left|\begin{matrix}
                    \textbf{x}&0&-a\\0&\textbf{x}-b&0\\-a&0&\textbf{x}
                \end{matrix}\right|
                \\ &= (\textbf{x}-b) \left| \begin{matrix}
                    \textbf{x} & -a \\ -a & \textbf{x}
                \end{matrix} \right|
                \\ &= (\textbf{x} - b)(\textbf{x}^2 - a^2)
                \\ &= (\textbf{x}-b)(\textbf{x}-a)(\textbf{x}+a) = c_A(x)
            \end{align*}

            \begin{itemize}
                \item[$\therefore$] $\lambda_1 = b; \lambda_2 = a; \lambda_3 = -a$
            \end{itemize}
            \begin{align*}
                (\lambda_1I-A)\textbf{x}_1 = 0 &\Leftrightarrow \begin{bmatrix}
                    b & 0 & -a \\ 0 & b-b&0\\-a&0&&b
                \end{bmatrix}\begin{bmatrix}
                    \textbf{x}_1^{(1)}\\\textbf{x}_1^{(2)}\\\textbf{x}_1^{(3)}
                \end{bmatrix} = \begin{bmatrix}
                    0&0&0
                \end{bmatrix}
                \\ b\textbf{x}_1^{(1)} - a\textbf{x}_1^{(3)} = 0 &\Rightarrow \textbf{x}_1^{(3)} = \frac{b}{a}\textbf{x}_1^{(3)}
                \\ -a\textbf{x}_1^{(1)}+b\textbf{x}_1^{(3)} = 0 &\Rightarrow \textbf{x}_1^{(3)} = \frac{a}{b}\textbf{x}_1^{(1)}
            \end{align*}

            \begin{itemize}
                \item[] Terapkan $a = \pm b$, ambil $a=-b$
                \item[] $\textbf{x}_1^{(1)}=1$ dan $\textbf{x}_1^{(2)}=1$
                \item[] Kita mendapat $\textbf{x}_1^{(3)}=1$ dan $\textbf{x}_1=(1,1,-1)$
            \end{itemize}
            \begin{align*}
                (\lambda_2I-A)\textbf{x}_2=0 &\Leftrightarrow \begin{bmatrix}
                    a&0&-a\\0&a-b&0\\-a&0&a
                \end{bmatrix}\begin{bmatrix}
                    \textbf{x}_2^{(1)} \\ \textbf{x}_2^{(2)} \\ \textbf{x}_2^{(3)}
                \end{bmatrix} = \begin{bmatrix}
                    0\\0\\0
                \end{bmatrix}
                \\ a\textbf{x}_2^{(1)} - a\textbf{x}_2^{(3)} = 0 &\Rightarrow \textbf{x}_2^{(1)} = \textbf{x}_2^{(3)}
                \\ (a-b)\textbf{x}_2^{(2)}= 0 &\Rightarrow -2\textbf{x}_2^{(2)} = 0 \Rightarrow \textbf{x}_2^{(2)}
                \\ a\textbf{x}_2^{(1)} - a\textbf{x}_2^{(3)} = 0 &\Rightarrow \textbf{x}_2^{(1)} = \textbf{x}_2^{(3)} 
            \end{align*}

            \begin{itemize}
                \item[] Terapkan $\textbf{x}_2^{(3)}=1$
                \item[] Kita mendapat $\textbf{x}_2^{(1)}=1$ dan $\textbf{x}_2=(1,0,1)$
            \end{itemize}
            \begin{align*}
                (\lambda_3I-A)\textbf{x}_3=0 &\Leftrightarrow \begin{bmatrix}
                    -a&0&-a\\0&-a-b&0\\-a&0&-a
                \end{bmatrix}\begin{bmatrix}
                    \textbf{x}_3^{(1)} \\ \textbf{x}_3^{(2)} \\ \textbf{x}_3^{(3)}
                \end{bmatrix} = \begin{bmatrix}
                    0\\0\\0
                \end{bmatrix}
                \\ a\textbf{x}_3^{(1)} - a\textbf{x}_3^{(3)} = 0 &\Rightarrow \textbf{x}_3^{(1)} = -\textbf{x}_3^{(3)}
                \\ (-a-b)\textbf{x}_3^{(2)}= 0 &\Rightarrow a=-b;\textbf{x}_3^{(2)} \in \mathbb{R}
                \\ -c\textbf{x}_3^{(2)} - k\textbf{x}_3^{(3)} = 0 &\Rightarrow \textbf{x}_3^{(1)} = -\textbf{x}_3^{(3)} 
            \end{align*}

            \begin{itemize}
                \item[] Terapkan $\textbf{x}_3^{(1)}=1$ dan $\textbf{x}_3^{(2)}=-2$
                \item[] Kita mendapat $\textbf{x}_3^{(3)}=1$ dan $\textbf{x}_3=(1,-2,-1)$
                \item[] Bukti vector orthogonal:
            \end{itemize}
            \begin{align*}
                \textbf{x}_1 \textbf{x}_2 &= (1,1,-1)(1,0,1)
                \\ &= 1-1 = 0
                \\ \textbf{x}_1 \textbf{x}_3 &= (1,1,-1) (1,-2,-1)
                \\ &= 1-2+1 = 0
                \\ \textbf{x}_2 \textbf{x}_3 &= (1,0,1)(1,-2,-1)
                \\ &= 1+0-1=0 
            \end{align*}

            \begin{itemize}
                \item[] Mencari norms dari eigenvectors:
            \end{itemize}
            \begin{align*}
                \Vert\textbf{x}_1\Vert &= \Vert(1,1,-1)\Vert
                \\ &= \sqrt{1^2+1^2+(-1)^2} = \sqrt{3}
                \\ \Vert\textbf{x}_2\Vert &= \Vert(1,0,-1)\Vert
                \\ &= \sqrt{1^2+0+1^2} = \sqrt{2}
                \\ \Vert\textbf{x}_3\Vert &= \Vert(1,-2,-1)\Vert
                \\ \sqrt{1^2 + (-2)^2 + (-1)^2} = \sqrt{6}
            \end{align*}

            \begin{itemize}
                \item[$\therefore$] Matriks P adalah:
            \end{itemize}
            \begin{align*}
                P &= \begin{bmatrix}
                    \frac{\textbf{x}_1}{\Vert\textbf{x}_1\Vert} & \frac{\textbf{x}_2}{\Vert\textbf{x}_2\Vert} & \frac{\textbf{x}_3}{\Vert\textbf{x}_3\Vert}
                \end{bmatrix}
                \\ &= \frac{\sqrt{6}}{6} \begin{bmatrix}
                    \sqrt{2} & \sqrt{3} & 1
                    \\ \sqrt{2} & 0 & =2
                    \\ -\sqrt{2} & \sqrt{3} & -1
                \end{bmatrix}
            \end{align*}

        \end{enumerate}

    \subsection*{Positive Definite Matrices}
        \begin{enumerate}
            \setcounter{enumi}{8}
            \item \textbf{Exercise 8.3.1}\\Find the Cholesky decomposition of each of the following matrices.
            \begin{enumerate}
                \setcounter{enumii}{3}
                \item $\begin{bmatrix}
                    20 & 4 & 5\\
                    4 & 2 & 3\\
                    5 & 3 & 5
                \end{bmatrix}$

                Diketahui:
                \begin{itemize}
                    \item[] $A=\begin{bmatrix}
                        20 & 4 & 5\\
                        4 & 2 & 3\\
                        5 & 3 & 5
                    \end{bmatrix}$
                \end{itemize}

                Ditanya:
                \begin{itemize}
                    \item Cholesky decomposition $A$?
                \end{itemize}

                Jawab:
                \begin{align*}
                    \mbox{det}(^{(1)}A)=20>0&;\mbox{det}(^{(2)}A)=24>0;\mbox{det}(^{(3)}A)=10>0
                    \\ A = &\begin{bmatrix}
                        20 & 4 & 5\\
                        4 & 2 & 3\\
                        5 & 3 & 5
                    \end{bmatrix}
                    \\ \textbf{R}_2 \to \textbf{R}_2 - \frac{\textbf{R}_1}{5} &\begin{bmatrix}
                        20 & 4 & 5\\
                        0 & \frac{6}{5} & 2\\
                        5 & 3 & 5
                    \end{bmatrix}
                    \\ \textbf{R}_3 \to \textbf{R}_3 - \frac{\textbf{R}_1}{4} &\begin{bmatrix}
                        20 & 4 & 5\\
                        0 & \frac{6}{5} & 2\\
                        0 & 2 & \frac{15}{4}
                    \end{bmatrix}
                    \\ \textbf{R}_3 \to \textbf{R}_3 - \frac{5\textbf{R}_2}{3} &\begin{bmatrix}
                        20 & 4 & 5\\
                        0 & \frac{6}{5} & 2\\
                        0 & 0 & \frac{5}{12}
                    \end{bmatrix} = V_1
                \end{align*}
                \begin{itemize}
                    \item[$\therefore$] $V = \begin{bmatrix}
                        2\sqrt{5} & \frac{2\sqrt{5}}{5} & \frac{\sqrt{5}}{2}
                        \\ 0 & \frac{\sqrt{30}}{5} & \frac{2\sqrt{30}}{6}
                        \\ 0 & 0 & \frac{\sqrt{60}}{12}
                    \end{bmatrix}$
                \end{itemize}

            \end{enumerate}
        \end{enumerate}

    \subsection*{QR-Factorization}
        \begin{enumerate}
            \setcounter{enumi}{9}
            \item \textbf{Exercise 8.4.1}\\In each case find the QR-factorization of $A$.
            \begin{enumerate}
                \setcounter{enumii}{3}
                \item $A = \begin{bmatrix}
                    1 & 1 & 0\\
                    -1 & 0 & 1\\
                    0 & 1 & 1\\
                    1 & -1 & 0
                \end{bmatrix}$

                Diketahui:
                \begin{itemize}
                    \item[] $A = \begin{bmatrix}
                        1 & 1 & 0\\
                        -1 & 0 & 1\\
                        0 & 1 & 1\\
                        1 & -1 & 0
                    \end{bmatrix}$
                \end{itemize}

                Ditanya:
                \begin{itemize}
                    \item QR-factorization dari $A$?
                \end{itemize}

                Jawab:
                \begin{align*}
                    \textbf{x}_1 &= \begin{bmatrix}
                        1&-1&0&1
                    \end{bmatrix}^T
                    \\ \textbf{x}_2 &= \begin{bmatrix}
                        1&0&1&-1
                    \end{bmatrix}^T
                    \\ \textbf{x}_3 &= \begin{bmatrix}
                        0&1&1&0
                    \end{bmatrix}^T
                \end{align*}
                \begin{itemize}
                    \item[] Gunakan Gram-Schmidt algorithm:
                \end{itemize}
                \begin{align*}
                    \textbf{f}_1&=\textbf{x}_1=\begin{bmatrix}
                        1&-1&0&1
                    \end{bmatrix}^T
                    \\\textbf{f}_1&=\textbf{x}_2-\frac{\textbf{x}_2\cdot\textbf{f}_1}{\Vert\textbf{f}_1\Vert^2}\textbf{f}_1
                    \\ &= \begin{bmatrix}
                        1&0&1&-1
                    \end{bmatrix}^T - 0 = \begin{bmatrix}
                        1&0&1&-1
                    \end{bmatrix}
                    \\ \textbf{f}_3 &= \textbf{x}_3-\frac{\textbf{x}_3\cdot\textbf{f}_1}{\Vert\textbf{f}_1\Vert^2}\textbf{f}_1-\frac{\textbf{x}_3\cdot\textbf{f}_2}{\Vert\textbf{f}_2\Vert^2}\textbf{f}_2
                    \\ &= \begin{bmatrix}
                        0&1&1&0
                    \end{bmatrix}^T - \frac{-1}{3}\begin{bmatrix}
                        1&-1&0&1
                    \end{bmatrix}^T - \frac{1}{3}\begin{bmatrix}
                        1&0&1&-1
                    \end{bmatrix}^T
                    \\ &= \frac{2}{3}\begin{bmatrix}
                        0&1&1&1
                    \end{bmatrix}^T
                \end{align*}
                \begin{itemize}
                    \item[] Menormalisasi:
                \end{itemize}
                \begin{align*}
                    \textbf{Q}_1=\frac{1}{\Vert\textbf{f}_1\Vert}\textbf{f}_1 &= \frac{\sqrt{3}}{3}\begin{bmatrix}
                        1&-1&0&1
                    \end{bmatrix}^T
                    \\ \textbf{Q}_2=\frac{1}{\Vert\textbf{f}_2\Vert}\textbf{f}_2 &= \frac{\sqrt{3}}{3}\begin{bmatrix}
                        1&0&1&-1
                    \end{bmatrix}^T
                    \\ \textbf{Q}_3=\frac{1}{\Vert\textbf{f}_3\Vert}\textbf{f}_3 &= \frac{\sqrt{3}}{3}\begin{bmatrix}
                        0&1&1&1
                    \end{bmatrix}^T
                \end{align*}
                \begin{itemize}
                    \item[$\therefore$] $\textbf{Q} = \frac{\sqrt{3}}{3}\begin{bmatrix}
                            1&1&0\\-1&0&1\\0&1&1\\1&-1&1
                        \end{bmatrix}$
                    \item[$\therefore$] $R=\frac{\sqrt{3}}{3}\begin{bmatrix}
                        3&0&-1\\0&3&1\\0&0&2
                    \end{bmatrix}$ 
                \end{itemize}
            \end{enumerate}
        \end{enumerate}

    \subsection*{Computing Eigenvalues}
        \begin{enumerate}
            \setcounter{enumi}{10}
            \item \textbf{Exercise 8.5.1}\\In each case, find the exact eigenvalues and determine corresponding eigenvectors. Then start with $\textbf{x}_0=\begin{bmatrix}
                1\\1
            \end{bmatrix}$ and compute $\textbf{x}_4$ and $r_3$ using the power method.
            \begin{enumerate}
                \item $A=\begin{bmatrix}
                    2&-4\\-3&3
                \end{bmatrix}$

                Diketahui:
                \begin{itemize}
                    \item[] $A=\begin{bmatrix}
                        2&-4\\-3&3
                    \end{bmatrix}$
                    \item[] $\textbf{x}_0=\begin{bmatrix}
                        1\\1
                    \end{bmatrix}$
                \end{itemize}

                Ditanya:
                \begin{itemize}
                    \item Eigenvalues, eigenvectors, $\textbf{x}_4$, dan $r_3$?
                \end{itemize}

                Jawab:
                \begin{align*}
                    (A-\textbf{x}I) &= \begin{bmatrix}
                        2-\lambda&-4\\-3\\3-\lambda
                    \end{bmatrix}
                    \\ &=(2-\lambda)(3-\lambda)-(4)(3)
                    \\ &=\lambda^2-5\lambda-6
                    \\ &=(\lambda+1)(\lambda-6)=0
                    \\ \therefore \lambda_1=-1 &\lor \lambda_2=6 
                \end{align*}
                
                \pagebreak

                \begin{itemize}
                    \item[] $\lambda_1 = -1$
                \end{itemize}
                \begin{align*}
                    \begin{bmatrix}
                        3&-4\\-3&4 
                    \end{bmatrix}\begin{bmatrix}
                        \textbf{x}_1\\\textbf{x}_2
                    \end{bmatrix}&=\begin{bmatrix}
                        0\\0
                    \end{bmatrix}
                    \\ 3\textbf{x}_1-4\textbf{x}_2&=0
                    \\ -3\textbf{x}_1+4\textbf{x}_2&=0
                    \\ \textbf{x}_1 &= \frac{4\textbf{x}_2}{3}
                    \\ \textbf{x}_2 &= \textbf{x}_2
                    \\ \therefore \mbox{Eigenvector = }\textbf{x}_2&\begin{bmatrix}
                        \frac{4}{3}\\1
                    \end{bmatrix}
                \end{align*}
                \begin{itemize}
                    \item[] $\lambda_2 = 6$
                \end{itemize}
                \begin{align*}
                    \begin{bmatrix}
                        4&4\\3&3
                    \end{bmatrix}\begin{bmatrix}
                        \textbf{x}_1\\\textbf{x}_2
                    \end{bmatrix} &= \begin{bmatrix}
                        0\\0
                    \end{bmatrix}
                    \\ 4\textbf{x}_1+4\textbf{x}_2&=0
                    \\ 3\textbf{x}_1+3\textbf{x}_2&=0
                    \\ \textbf{x}_1 &= -\textbf{x}_2
                    \\ \textbf{x}_2 &= \textbf{x}_2
                    \\ \therefore \mbox{Eigenvector = }\textbf{x}_2&\begin{bmatrix}
                        -1\\1
                    \end{bmatrix}
                \end{align*}
                \begin{align*}
                    \textbf{x}_0 &= \begin{bmatrix}
                        1\\1
                    \end{bmatrix}
                    \\ \textbf{x}_1 = A\textbf{x}_0 &= \begin{bmatrix}
                        -2\\0
                    \end{bmatrix}
                    \\ \textbf{x}_2 = A\textbf{x}_1 &= \begin{bmatrix}
                        -4\\6
                    \end{bmatrix}
                    \\ \textbf{x}_3 = A\textbf{x}_2 &= \begin{bmatrix}
                        -32\\30
                    \end{bmatrix}
                    \\ \therefore \textbf{x}_4 = A\textbf{x}_3 &= \begin{bmatrix}
                        -184\\186
                    \end{bmatrix}
                \end{align*}
                \begin{align*}
                    \therefore r_3&=\frac{\textbf{x}_k\cdot\textbf{x}_{k+1}}{\Vert\textbf{x}_k\Vert^2}
                    \\ &= \frac{\textbf{x}_3\cdot\textbf{x}_4}{\Vert\textbf{x}_3\Vert^2}
                    \\ &= \frac{2867}{481}
                \end{align*}

                \pagebreak

                \item $A=\begin{bmatrix}
                    5&2\\-3&-2
                \end{bmatrix}$

                Diketahui:
                \begin{itemize}
                    \item[] $A=\begin{bmatrix}
                        5&2\\-3&-2
                    \end{bmatrix}$
                    \item[] $\textbf{x}_0=\begin{bmatrix}
                        1\\1
                    \end{bmatrix}$
                \end{itemize}

                Ditanya:
                \begin{itemize}
                    \item Eigenvalues, eigenvectors, $\textbf{x}_4$, dan $r_3$?
                \end{itemize}

                Jawab:
                \begin{align*}
                    (A-\textbf{x}I) &= \begin{bmatrix}
                        5-\lambda&2\\-3\\-2-\lambda
                    \end{bmatrix}
                    \\ &=(5-\lambda)(2-\lambda)-(-3)(2)
                    \\ &=\lambda^2-3\lambda-4
                    \\ &=(\lambda+1)(\lambda-4)=0
                    \\ \therefore \lambda_1=-1 &\lor \lambda_2=4 
                \end{align*}
                \begin{itemize}
                    \item[] $\lambda_1 = -1$
                \end{itemize}
                \begin{align*}
                    &\begin{bmatrix}
                        6&2\\-3&3    
                    \end{bmatrix}
                    \\ \textbf{R}_1 \to \frac{\textbf{R}_1}{6} &\begin{bmatrix}
                        1&\frac{1}{3}\\-3&-1
                    \end{bmatrix}
                    \\ \textbf{R}_2 \to \textbf{R}_2 - 3\textbf{R}_1 &\begin{bmatrix}
                        1&\frac{1}{3}\\0&0
                    \end{bmatrix}
                    \\ \textbf{x}_1 &= -\frac{\textbf{x}_2}{3}
                    \\ \textbf{x}_2 &= \textbf{x}_2
                    \\ \therefore \mbox{Eigenvector = }\textbf{x}_2&\begin{bmatrix}
                        -\frac{1}{3}\\1
                    \end{bmatrix}
                \end{align*}

                \pagebreak

                \begin{itemize}
                    \item[] $\lambda_2 = 4$
                \end{itemize}
                \begin{align*}
                    &\begin{bmatrix}
                        1&2\\-3&-6
                    \end{bmatrix}
                    \\ \textbf{R}_2 \to \textbf{R}_2 + 3\textbf{R}_1 &\begin{bmatrix}
                        1&2\\0&0
                    \end{bmatrix}
                    \\ \textbf{x}_1&=-2\textbf{x}_2
                    \\ \textbf{x}_2&=\textbf{x}_1
                    \\ \therefore \mbox{Eigenvector = }\textbf{x}_2&\begin{bmatrix}
                        -2\\1
                    \end{bmatrix}
                \end{align*}
                \begin{align*}
                    \textbf{x}_0 &= \begin{bmatrix}
                        1\\1
                    \end{bmatrix}
                    \\ \textbf{x}_1 = A\textbf{x}_0 &= \begin{bmatrix}
                        7\\-5
                    \end{bmatrix}
                    \\ \textbf{x}_2 = A\textbf{x}_1 &= \begin{bmatrix}
                        25\\-11
                    \end{bmatrix}
                    \\ \textbf{x}_3 = A\textbf{x}_2 &= \begin{bmatrix}
                        103\\-53
                    \end{bmatrix}
                    \\ \therefore\textbf{x}_4 = A\textbf{x}_3 &= \begin{bmatrix}
                        409\\-203
                    \end{bmatrix}
                \end{align*}
                \begin{align*}
                    \therefore r_3&=\frac{\textbf{x}_k\cdot\textbf{x}_{k+1}}{\Vert\textbf{x}_k\Vert^2}
                    \\ &= \frac{\textbf{x}_3\cdot\textbf{x}_4}{\Vert\textbf{x}_3\Vert^2}
                    \\ &= \frac{52886}{13418}
                \end{align*}

            \end{enumerate}
        \end{enumerate}

    \subsection*{The Singular Value Decomposition}
        \begin{enumerate}
            \setcounter{enumi}{11}
            \item \textbf{Exercise 8.6.8}\\Let $A^{-1}=A=A^T$ where $A$ is $n \times n$. Given any orthogonal $n \times n$ matrix $U$, find an orthogonal matrix $V$ such that $A=U\Sigma _AV^T$ is an SVD for $A$.\\If $A=\begin{bmatrix}
                0&1\\1&0
            \end{bmatrix}$ do this for:
            \begin{enumerate}
                \item $U=\frac{1}{5}\begin{bmatrix}
                    3&-4\\4&3
                \end{bmatrix}$

                \pagebreak

                Diketahui:
                \begin{itemize}
                    \item[] $A^{-1}=A=A^T$
                    \item[] $A=U\Sigma _AV^T$ adalah SVD dari $A$
                    \item[] $U=\frac{1}{5}\begin{bmatrix}
                        3&-4\\4&3
                    \end{bmatrix}$
                \end{itemize}

                Ditanya:
                \begin{itemize}
                    \item Matriks orthogonal $V$?
                \end{itemize}

                Jawab:
                \begin{align*}
                    A^T\cdot A&=\begin{bmatrix}
                        0&1\\1&0
                    \end{bmatrix}\begin{bmatrix}
                        0&1\\1&0
                    \end{bmatrix}=\begin{bmatrix}
                        1&0\\0&1
                    \end{bmatrix}
                    \\ (A-\textbf{x}I)&=\begin{bmatrix}
                        1-\lambda&0\\0&1-\lambda
                    \end{bmatrix}
                    \\ (1-\lambda)^2&=0
                    \\ \lambda_1=1&\lor \lambda_2=1
                \end{align*}
                \begin{itemize}
                    \item[] $\lambda = 1$
                \end{itemize}
                \begin{align*}
                    &\begin{bmatrix}
                        0&0\\0&0
                    \end{bmatrix}
                    \\ \textbf{x}_1&=\textbf{x}_1
                    \\ \textbf{x}_2&=\textbf{x}_2
                    \\ \textbf{x}_1\begin{bmatrix}
                        1\\0
                    \end{bmatrix} &+ \textbf{x}_2\begin{bmatrix}
                        0\\1
                    \end{bmatrix}
                \end{align*}
                \begin{itemize}
                    \item[$\therefore$] Eigenvalues adalah $\lambda_1=1, \lambda_2=1$ dan eigenvectors adalah $\left\{\begin{bmatrix}
                        1\\0
                    \end{bmatrix},\begin{bmatrix}
                        0\\1
                    \end{bmatrix}\right\}$ 
                \end{itemize}
                \begin{align*}
                    \sigma_1&=1; \sigma_2=1
                    \\ \Sigma_A&=\begin{bmatrix}
                        1&0\\0&1
                    \end{bmatrix}
                    \\V^{T-1}&=A^{-1}\lor\Sigma_A
                    \\ &= \begin{bmatrix}
                        0&1\\1&0
                    \end{bmatrix}\begin{bmatrix}
                        \frac{3}{5}&-\frac{4}{5}\\\frac{4}{5}&\frac{3}{5}
                    \end{bmatrix}\begin{bmatrix}
                        1&0\\0&1
                    \end{bmatrix}
                    \\ &= \frac{1}{5}\begin{bmatrix}
                        4&3\\3&-4
                    \end{bmatrix}
                    \\ V^T&=\begin{bmatrix}
                        -4&-3\\-3&4
                    \end{bmatrix}
                    \\ \therefore V &= \begin{bmatrix}
                        -4&-3\\-3&4
                    \end{bmatrix}
                \end{align*}

            \end{enumerate}
            
            \item \textbf{Exercise 8.6.9}\\Find an SVD for the following matrices:
            \begin{enumerate}
                \setcounter{enumii}{1}
                \item $\begin{bmatrix}
                    1 & 1 & 1\\
                    -1 & 0 & -2\\
                    1 & 2 & 0
                \end{bmatrix}$
                
                Diketahui:
                \begin{itemize}
                    \item[] $A=\begin{bmatrix}
                        1 & 1 & 1\\
                        -1 & 0 & -2\\
                        1 & 2 & 0
                    \end{bmatrix}$
                \end{itemize}

                Ditanya:
                \begin{itemize}
                    \item SVD dari $A$?
                \end{itemize}

                Jawab:
                \begin{align*}
                    A&=\begin{bmatrix}
                        1 & 1 & 1\\
                        -1 & 0 & -2\\
                        1 & 2 & 0
                    \end{bmatrix}
                    \\A^T&=\begin{bmatrix}
                        1 & -1 & 1\\
                        1 & 0 & 2\\
                        1 & -2 & 0
                    \end{bmatrix}
                    \\A^TA&=\begin{bmatrix}
                        3&3&3\\3&5&1\\3&1&5
                    \end{bmatrix}
                    \\ (A-\textbf{x}I) &= \begin{bmatrix}
                        3-\lambda&3&3\\3&5-\lambda&1\\3&1&5-\lambda
                    \end{bmatrix}
                    \\ \lambda_1=0\lor \lambda_2 &= 4 \lor \lambda_3=9
                \end{align*}
                \begin{itemize}
                    \item[] $\lambda_1=0$
                \end{itemize}
                \begin{align*}
                    \begin{bmatrix}
                        3&3&3\\3&5&1\\3&1&5
                    \end{bmatrix} &\to \begin{bmatrix}
                        1&0&2\\0&1&-1\\0&0&0
                    \end{bmatrix}
                    \\ \textbf{x}_3\begin{bmatrix}
                        -2\\1\\1
                    \end{bmatrix}
                \end{align*}
                \begin{itemize}
                    \item[] $\lambda_2=4$
                \end{itemize}
                \begin{align*}
                    \begin{bmatrix}
                        -1&3&3\\3&1&1\\3&1&1
                    \end{bmatrix} &\to \begin{bmatrix}
                        1&0&0\\0&1&1\\0&0&0
                    \end{bmatrix}
                    \\ \textbf{x}_3\begin{bmatrix}
                        0\\-1\\1
                    \end{bmatrix}
                \end{align*}
                \begin{itemize}
                    \item[] $\lambda_3=9$
                \end{itemize}
                \begin{align*}
                    \begin{bmatrix}
                        -6&3&3\\3&-4&1\\3&1&-4
                    \end{bmatrix} &\to \begin{bmatrix}
                        1&0&-1\\0&1&-1\\0&0&0
                    \end{bmatrix}
                    \\ \textbf{x}_3\begin{bmatrix}
                        1\\1\\1
                    \end{bmatrix}
                \end{align*}
                \begin{itemize}
                    \item[$\therefore$] Eigenvalues adalah $\lambda_1=0,\lambda_2=4,\lambda_3=9$ dan eigenvectors adalah $\left\{\begin{bmatrix}
                        -2\\1\\1
                    \end{bmatrix},\begin{bmatrix}
                        0\\-1\\1
                    \end{bmatrix},\begin{bmatrix}
                        1\\1\\1
                    \end{bmatrix}\right\}$.
                \end{itemize}
                \begin{align*}
                    \sigma_1&=\sqrt{9}=3
                    \\ \sigma_2&=\sqrt{4}=2
                    \\ \Sigma_A&=\begin{bmatrix}
                        3&0&0\\0&2&0\\0&0&0
                    \end{bmatrix}
                    \\ q_1 = \begin{bmatrix}
                        \frac{\sqrt{3}}{3}\\\frac{\sqrt{3}}{3}\\\frac{\sqrt{3}}{3}
                    \end{bmatrix}
                    , q_2 &= \begin{bmatrix}
                        0\\-\frac{\sqrt{2}}{2}\\\frac{\sqrt{2}}{2}
                    \end{bmatrix}
                    , q_3 = \begin{bmatrix}
                        -\frac{\sqrt{6}}{3}\\\frac{\sqrt{6}}{6}\\\frac{\sqrt{6}}{6}
                    \end{bmatrix}
                    \\ \therefore Q &= \begin{bmatrix}
                        \frac{\sqrt{3}}{3}&0&-\frac{\sqrt{6}}{3}\\\frac{\sqrt{3}}{3}&-\frac{\sqrt{2}}{2}&\frac{\sqrt{6}}{6}\\\frac{\sqrt{3}}{3}&\frac{\sqrt{2}}{2}&\frac{\sqrt{6}}{6}
                    \end{bmatrix}
                    \\ P_1 = \begin{bmatrix}
                        \frac{\sqrt{3}}{9}\\\frac{\sqrt{3}}{3}\\-\frac{\sqrt{3}}{9}
                    \end{bmatrix}
                    , P_2 &= \begin{bmatrix}
                        0\\\frac{\sqrt{2}}{2}\\-\frac{\sqrt{2}}{2}
                    \end{bmatrix}
                    , P_3 = \begin{bmatrix}
                        -\frac{\sqrt{6}}{3}\\\frac{\sqrt{6}}{6}\\\frac{\sqrt{6}}{6}
                    \end{bmatrix}
                    \\ \therefore P &= \begin{bmatrix}
                        \frac{\sqrt{3}}{9}&0&-\frac{\sqrt{6}}{3}\\\frac{\sqrt{3}}{3}&\frac{\sqrt{2}}{2}&\frac{\sqrt{6}}{6}\\-\frac{\sqrt{3}}{9}&-\frac{\sqrt{2}}{2}&\frac{\sqrt{6}}{6}
                    \end{bmatrix}
                    \\ \therefore A &= P\Sigma_AQ^T
                    \\ &=\begin{bmatrix}
                        \frac{\sqrt{3}}{9}&0&-\frac{\sqrt{6}}{3}\\\frac{\sqrt{3}}{3}&\frac{\sqrt{2}}{2}&\frac{\sqrt{6}}{6}\\-\frac{\sqrt{3}}{9}&-\frac{\sqrt{2}}{2}&\frac{\sqrt{6}}{6}
                    \end{bmatrix}\begin{bmatrix}
                        3&0&0\\0&2&0\\0&0&0
                    \end{bmatrix}\begin{bmatrix}
                        \frac{\sqrt{3}}{3}&\frac{\sqrt{3}}{3}&\frac{\sqrt{3}}{3}\\0&-\frac{\sqrt{2}}{2}&\frac{\sqrt{2}}{2}\\-6\sqrt{3}&\frac{\sqrt{6}}{6}&\frac{\sqrt{6}}{6}
                    \end{bmatrix}
                \end{align*}

            \end{enumerate}
        \end{enumerate}

    \pagebreak

    \subsection*{Complex Matrices}
        \begin{enumerate}
            \setcounter{enumi}{13}
            \item \textbf{Exercise 8.7.2}\\In each case, determine whether the two vectors are orthogonal.
            \begin{enumerate}
                \setcounter{enumii}{1}
                \item $(i, -i, 2+i),\mbox{ }(i, i, 2-i)$
                
                Diketahui:
                \begin{itemize}
                    \item[] $V=(i, -i, 2+i)$
                    \item[] $U=(i, i, 2-i)$
                \end{itemize}

                Ditanya:
                \begin{itemize}
                    \item Apakah kedua vector orthogonal?
                \end{itemize}

                Jawab:
                \begin{itemize}
                    \item[] Vector orthogonal apabila $V \cdot U = 0$
                \end{itemize}
                \begin{align*}
                    (i, -i, 2+i)\cdot(i, i, 2-i)&=i^2-i^2+(2+i)(2-i)
                    \\ &= -2i^2 \neq 0
                \end{align*}

                \begin{itemize}
                    \item[$\therefore$] Kedua vector tersebut tidak orthogonal karena $V \cdot U \neq 0$.
                \end{itemize}
                
            \end{enumerate}
            
            \item \textbf{Exercise 8.7.8}\\In each case, find a unitary matrix $U$ such that $U^HAU$ is diagonal.
            \setcounter{enumii}{5}
            \begin{enumerate}
                \item $A=\begin{bmatrix}
                    1&0&0\\
                    0&1&1+i\\
                    0&1-i&2
                \end{bmatrix}$

                Diketahui:
                \begin{itemize}
                    \item[] $A=\begin{bmatrix}
                        1&0&0\\
                        0&1&1+i\\
                        0&1-i&2
                    \end{bmatrix}$
                    \item[] $U^HAU$ diagonal
                \end{itemize}

                Ditanya:
                \begin{itemize}
                    \item Matriks unitary $U$?
                \end{itemize}

                Jawab:
                \begin{align*}
                    (\textbf{x}I-A) &= \begin{bmatrix}
                        \textbf{x}-1&0&0\\0&\textbf{x}-1&-1-i\\0&-1-i&x-2
                    \end{bmatrix}
                    \\ (\textbf{x}-1)(\textbf{x}^2-3\textbf{x})&=(\textbf{x}-1)(\textbf{x}-3)
                    \\ \textbf{x}=1\lor \textbf{x}&=0 \lor \textbf{x}=3
                \end{align*}
                \begin{itemize}
                    \item[$\therefore$] Eigenvalues adalah $\lambda_1=0, \lambda_2=1,\lambda_3=3$
                    \pagebreak
                    \item[] $\lambda_1 = 0$
                \end{itemize}
                \begin{align*}
                    \begin{bmatrix}
                        =1&0&0\\0&-1&-1-i\\0&-1+i&-1
                    \end{bmatrix}&\to\begin{bmatrix}
                        0&1&0\\0&0&1\\0&0&0
                    \end{bmatrix}
                    \\\textbf{x}_1&=\textbf{x}_1
                    \\\textbf{x}_2&=0
                    \\\textbf{x}_3&=0
                    \\\textbf{x}_1&\begin{bmatrix}
                        1\\0\\0
                    \end{bmatrix}
                \end{align*}
                \begin{itemize}
                    \item[] $\lambda_2=3$
                \end{itemize}
                \begin{align*}
                    \begin{bmatrix}
                        2&0&0\\0&2&-1-i\\0&-1+i&1
                    \end{bmatrix} &\to \begin{bmatrix}
                        1&0&0\\0&-1+i&1\\0&0&0
                    \end{bmatrix}
                    \\ \textbf{x}_1&=0
                    \\ \textbf{x}_2&=\textbf{x}_2
                    \\ \textbf{x}_3&=1-i(\textbf{x}_2)
                    \\ \textbf{x}_2&\begin{bmatrix}
                        0&1&1-i
                    \end{bmatrix}
                \end{align*}
                \begin{align*}
                    \therefore U &=\begin{bmatrix}
                        1&0&0\\0&\frac{\sqrt{3}+i\sqrt{3}}{3}&\frac{\sqrt{3}}{3}\\0&-\frac{\sqrt{3}}{3}&\frac{\sqrt{3}-i\sqrt{3}}{3}
                    \end{bmatrix} = \frac{\sqrt{3}}{3}\begin{bmatrix}
                        1&0&0\\0&1+i&1\\0&-1&1-i
                    \end{bmatrix}; U \mbox{ orthogonal}
                    \\ U^HAU&=\begin{bmatrix}
                        1&0&0\\0&0&0\\0&0&3
                    \end{bmatrix}
                \end{align*}

            \end{enumerate}
        \end{enumerate}

\end{document}
